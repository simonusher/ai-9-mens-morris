\documentclass{article}
\usepackage{polski}
\usepackage{graphicx}
\usepackage{amsmath}
\usepackage{hyperref}
\usepackage{float}

\hypersetup{%
	pdfborder = {0 0 0}
}

\author{Szymon Woźniak, 235040}
\date{16.05.2019}
\title{Algorytmy rozwiązywania gier o sumie zerowej}


\begin{document}
	\pagenumbering{gobble}
	\maketitle
	\newpage
	\pagenumbering{arabic}
	
	\section{Wstęp teoretyczny}
	\subsection{Gra planszowa Młynek}
	\begin{figure}[H]
		\centering
		\includegraphics[width=0.7\linewidth]{mill_board.png}
		\caption{Plansza do gry w młynek z kilkoma rozstawionymi pionkami}
		\label{fig:board}
	\end{figure}
	\subsubsection{Skrót zasad}
	Młynek jest dwuosobową, turową, logiczną grą planszową. W rozpatrywanej wersji, na planszy znajdują się 24 rozmieszczone na 3 koncentrycznych kwadratach pola. Na każdym z tych pól gracze mogą umieszczać swoje pionki. Obaj gracze posiadają po 9 pionków do rozmieszczenia.\\
	Gracze poprzez odpowiednie rozstawianie swoich pionków, mogą blokować lub zbijać pionki przeciwników. Bicie następuje gdy jeden z graczy ustawi 3 swoje pionki w linię. Może wtedy wybrać jeden z pionków przeciwnika, który zostanie usunięty z planszy.\\
	\subsubsection{Fazy rozgrywki}
	Pojedyncza partia młynka składa się z trzech faz.
	\begin{itemize}
		\item rozstawianie pionków,
		\item przesuwanie pionków,
		\item "latanie".
	\end{itemize}
	W pierwszej fazie rozgrywki gracze na zmianę umieszczają po jednym z dostępnych 9 pionków na wolnych polach planszy. Jeżeli któremuś z nich uda się ustawić młynek, może usunąć z planszy wybrany pionek przeciwnika. W rozpatrywanej wersji gry, jest to jedyny moment kiedy gracz może ustawić podwójny młynek.\\
	W drugiej, podstawowej fazie rozgrywki gracze na zmianę swoje pionki. Mogą wybrać dowolne puste pole połączone linią z polem na którym znajduje się aktualnie pionek.\\
	Trzecia faza następuje dla każdego gracza osobno, kiedy pozostaną mu tylko 3 pionki. Może on wtedy w swojej turze przemieszczać pionki na dowolne puste miejsca na planszy (stąd angielska nazwa \textit{flying}).
	\subsubsection{Cel rozgrywki}
	Celem rozgrywki jest doprowadzenie do sytuacji, w której przeciwnikowi pozostaną tylko 2 pionki, lub nie posiada on żadnego możliwego ruchu.
	\subsubsection{Dodatkowe modyfikacje}
	W rozpatrywanej wersji gry stosuje się zasadę, że pionka nie można przesunąć na pole, z którego został przesunięty wcześniej. Tym samym gracze zmuszeni są budować młynki, zamiast korzystać z już istniejących.\\
	W niektórych wersjach gry stosuje się też zasadę, że gracze mogą zbijać pionki przeciwników tylko pod warunkiem, że nie stoją w młynku. W tej pracy zasada ta nie została zastosowana.
	\subsection{Badane algorytmy}
	\subsubsection{Algorytm min-max}
	\subsubsection{Algorytm alfa-beta cięć}
	\section{Plan pracy}
	
	\section{Heurystyki oceny stanu planszy}
	\subsection{Liczba pionków}
	\subsection{Liczba pionków i liczba młynków}
	\subsection{Liczba pionków i liczba możliwych ruchów}
	\section{Badania}
	\section{Podsumowanie}
\end{document}